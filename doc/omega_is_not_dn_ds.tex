\documentclass[a4paper,oneside,12pt]{article}
\usepackage[a4paper,includehead,includefoot,bindingoffset=2cm,margin=2cm]{geometry}

\usepackage[dvipsnames,usenames]{color}

%\usepackage{epsfig,fancybox,multicol,pstcol}
%\usepackage{subfig}

\usepackage{amsmath,amssymb, amsthm}
\usepackage{mathrsfs}
\usepackage{psfrag}
\usepackage{graphicx}
\usepackage{scalefnt}
\usepackage{enumerate}
\usepackage[round]{natbib}
%\usepackage{hyphenat}
\usepackage{helvet}
\usepackage{ulem}
\usepackage{pdflscape}
\usepackage{algorithmic}
\usepackage[section]{algorithm}
\usepackage[doublespacing]{setspace}

\bibpunct{(}{)}{;}{a}{}{,}

\numberwithin{equation}{section}

%\parindent = 0 cm
%\parskip = 12 pt
%\columnwidth = 23 pc
%\addtolength{\oddsidemargin}{-.5in}
%\addtolength{\evensidemargin}{-.5in}
%\addtolength{\textwidth}{1in}

%\addtolength{\topmargin}{-.5in}
%\addtolength{\textheight}{1in}

%%
% Various useful definitions
%%
\def\var{\mathop{\rm var}\nolimits}    %variance
\def\cov{\mathop{\rm cov}\nolimits}    %covariance
\def\corr{\mathop{\rm corr}\nolimits}  %correlation
%
\newcommand{\rd}{\mathrm{d}} %differential d
\DeclareMathOperator\sd{d\!} %differential d preceeded by a thin space, when needed
\newcommand{\ri}{\mathrm{i}} %imaginary i
\newcommand{\re}{\mathrm{e}} %exponential e
\DeclareMathOperator\sP{P}   %probability, treated as an operator
\newcommand{\rP}{\mathrm{P}} %probability, when spacing isn't a worry
\DeclareMathOperator\sE{E}   %expectation, treated as an operator
\newcommand{\rE}{\mathrm{E}} %expectation, when spacing isn't a worry
\DeclareMathOperator\BigO{O} %Big-0
\DeclareMathOperator\littleo{o} %little-o
\DeclareMathOperator\diag{diag}
%
\newcommand{\trans}{^\top}   %matrix transpose
%
\newcommand{\cvgpr}{\xrightarrow{\text{\upshape\tiny P}}}   %convergence in probability
\newcommand{\cvgas}{\xrightarrow{\text{\upshape\tiny a.s.}}}   %convergence almost surely
\newcommand{\cvgwk}{\xrightarrow{\text{\upshape\tiny W}}}   %weak convergence
\newcommand{\cvgdist}{\xrightarrow{\text{\upshape\tiny D}}} %convergence in distribution
\newcommand{\eqdist}{\stackrel{\text{\upshape\tiny D}}{=}}  %equality in distribution
\newcommand{\cvgs}{\stackrel{\text{\upshape\tiny S}}{\implies}}
\newcommand{\cvgmz}{\stackrel{\text{\upshape\tiny MZ}}{\implies}}

\DeclareMathOperator{\sgn}{sgn}
\DeclareMathOperator{\expint}{E_1}
\DeclareMathOperator{\Norm}{N}
\DeclareMathOperator{\order}{O}
\DeclareMathOperator{\ind}{\mathbf{1}}
\DeclareMathOperator{\besselK}{K}
\newcommand{\reals}{\mathbb{R}}
\newcommand{\plus}{{\mathord{+}}}
\newcommand{\minus}{{\mathord{-}}}
\newcommand{\naturalnumbers}{\mathbb{N}}
\newcommand{\integers}{\mathbb{Z}}
\newcommand{\esssup}{\mathop{\rm ess\,sup}}
\newcommand{\argmin}{\mathop{\rm arg\,min}}
\newcommand{\argmax}{\mathop{\rm arg\,max}}

\def\combin#1#2{\left(\begin{array}{c}\!\!{#1}\!\!\\\!\!{#2}\!\!\end{array}\right)}

\makeatletter
\def\imod#1{\allowbreak\mkern10mu({\operator@font mod}\,\,#1)}
\makeatother

\newtheorem{thm}{Theorem}[section]
%\newtheorem{cor}[thm]{Corollary}
%\newtheorem{lem}[thm]{Lemma}
%\newtheorem{prop}[thm]{Proposition}
%\newtheorem{conj}[thm]{Conjecture}
%\newtheorem{remark}[thm]{Remark}
\newtheorem{lem}{Lemma}[section]
\newtheorem{remark}{Remark}[section]

%\newenvironment{remark}[1][Remark]{\begin{trivlist}
%\item[\hskip \labelsep {\bfseries #1}]}{\end{trivlist}}

\newcounter{enumtracker}
\newcounter{enumtrackerone}

% Centré, avec un signe devant le numéro de appendix
\makeatletter
\def\appendix{\renewcommand{\thesection}{\thechapter.\Alph{section}}
	\@ifstar\unnumberedappendix\numberedappendix}
\def\numberedappendix{\@ifnextchar[%]
  \numberedappendixwithtwoarguments\numberedappendixwithoneargument}
\def\unnumberedappendix{\@ifnextchar[%]
  \unnumberedappendixwithtwoarguments\unnumberedappendixwithoneargument}
\def\numberedappendixwithoneargument#1{\numberedappendixwithtwoarguments[#1]{#1}}
\def\unnumberedappendixwithoneargument#1{\unnumberedappendixwithtwoarguments[#1]{#1}}
\def\numberedappendixwithtwoarguments[#1]#2{%
  \ifhmode\par\fi
  \removelastskip
  \vskip 2ex\goodbreak
  \refstepcounter{section}%
  \begingroup
  \noindent\leavevmode\Large\bfseries\raggedright 
  Appendix\ \thesection\quad#2\par
  \endgroup
  \vskip 1ex\nobreak
  \addcontentsline{toc}{section}{%
    \protect\numberline{\thesection}%
    #1}%
  }
\def\unnumberedappendixwithtwoarguments[#1]#2{%
  \ifhmode\par\fi
  \removelastskip
  \vskip 2ex\goodbreak
%  \refstepcounter{section}%
  \begingroup
  \noindent\leavevmode\Large\bfseries\raggedright 
%  \S \thesection\ 
  #2\par
  \endgroup
  \vskip 1ex\nobreak
  \addcontentsline{toc}{section}{%
%    \protect\numberline{\thesection}%
    #1}%
  }
\makeatother



\begin{document}
\title{Some Notes on $\omega$}
\author{Ben Kaehler}
%\date{26 March 2012}
\maketitle
\subsection*{$\omega$ is not $d_\text{N}/d_\text{S}$}
If we define the instantaneous transition rates of a codon substitution
process as
\begin{align*}
q_{i,j} = \begin{cases} 0, &\text{codons differ at more than one position},\\
r_{i,j}, &\text{one synonymous difference between codons},\\
\omega r_{i,j}, &\text{one non-synonymous difference between codons},\end{cases}
\end{align*}
where $r_{i,j}$ is some representation of an evolutionarily neutral
process and $\omega$ summarises the influence of selective pressure,
then $\omega$ is not the ratio of non-synonymous to synonymous
substitutions. Note that $r_{i,j}$ is allowed to vary by transition in
this formulation but is usually further constrained to enforce, for
instance, time-reversibility of the process.

Define two new transition matrices:
\begin{align*}
q_{i,j}^\text{S} &= \begin{cases}
r_{i,j}, &\text{one synonymous difference between codons},\\
0, &\text{otherwise},\end{cases} \\
\text{and} \\
q_{i,j}^\text{N} &= \begin{cases}
r_{i,j}, &\text{one non-synonymous difference between codons},\\
0, &\text{otherwise},\end{cases}
\end{align*}
so that $Q = \omega Q^\text{N} + Q^\text{S}$, where $Q$, $Q^\text{N}$, and $Q^\text{S}$ are the matrices comprised of $q_{i,j}$, $q_{i,j}^\text{N}$, and $q_{i,j}^\text{S}$, respectively. Given an initial codon probability row-vector $\pi_0$, the expected number of substitutions up to time $t$ is
\begin{align*}
\mathrm{ENS} &= -\pi_0\int_0^t \re^{Qt} \rd t\diag Q \\
&= -\omega\pi_0\int_0^t \re^{Qt} \rd t\diag Q_\text{N} -\pi_0\int_0^t \re^{Qt} \rd t\diag Q_\text{S} \\
&=\mathrm{ENNS} + \mathrm{ENSS},
\end{align*}
where $\mathrm{ENNS}$ is the expected number of non-synonymous substitutions and $\mathrm{ENSS}$ is the expected number of synonymous substitutions. Now introduce the $\mathrm{ENNS}_1$ as the expected number of non-synonymous substitutions under neutral evolution ($\omega=1$) via
\begin{align*}
\mathrm{ENNS} = \omega\mathrm{ENNS}_1.
\end{align*}

For clarity write
\begin{align*}
\omega = \frac{\mathrm{ENS}-\mathrm{ENSS}}{\mathrm{ENNS}_1} = \frac{\mathrm{ENNS}}{\mathrm{ENNS}_1}.
\end{align*}
It is clear that, in general, $\omega$ is not the ratio of non-synonymous to synonymous substitutions. $\omega$ would be better described as the ratio of non-synonymous substitutions to non-synonymous substitutions under neutral evolution.

\subsection*{Combining $\omega$ estimates across branches}
Armed with this definition we are able to extend the definition of $\omega$ across branches of a phylogeny that are attributed with different models of codon evolution. Given two edges, $A$ and $B$, say that we know $\omega_i$, $\mathrm{ENNS}_i$, and $\mathrm{ENNS}_{1i}$ for $i\in\{A,B\}$. It is then natural to estimate $\omega$ across the two edges as
\begin{align*}
\omega_{A,B} = \frac{\mathrm{ENNS}_A+\mathrm{ENNS}_B}{\mathrm{ENNS}_{1A}+\mathrm{ENNS}_{1B}},
\end{align*}
thereby maintaining the definition of $\omega$ as the ratio of non-synonymous substitutions to non-synonymous substitutions under neutral evolution.
\end{document}
