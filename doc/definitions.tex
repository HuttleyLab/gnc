\usepackage[a4paper,includehead,includefoot,bindingoffset=2cm,margin=2cm]{geometry}

\usepackage[dvipsnames,usenames]{color}

%\usepackage{epsfig,fancybox,multicol,pstcol}
%\usepackage{subfig}

\usepackage{amsmath,amssymb, amsthm}
\usepackage{mathrsfs}
\usepackage{psfrag}
\usepackage{graphicx}
\usepackage{scalefnt}
\usepackage{enumerate}
\usepackage[round]{natbib}
%\usepackage{hyphenat}
\usepackage{helvet}
\usepackage{ulem}
\usepackage{pdflscape}
\usepackage{algorithmic}
\usepackage[section]{algorithm}
\usepackage[doublespacing]{setspace}

\bibpunct{(}{)}{;}{a}{}{,}

\numberwithin{equation}{section}

%\parindent = 0 cm
%\parskip = 12 pt
%\columnwidth = 23 pc
%\addtolength{\oddsidemargin}{-.5in}
%\addtolength{\evensidemargin}{-.5in}
%\addtolength{\textwidth}{1in}

%\addtolength{\topmargin}{-.5in}
%\addtolength{\textheight}{1in}

%%
% Various useful definitions
%%
\def\var{\mathop{\rm var}\nolimits}    %variance
\def\cov{\mathop{\rm cov}\nolimits}    %covariance
\def\corr{\mathop{\rm corr}\nolimits}  %correlation
%
\newcommand{\rd}{\mathrm{d}} %differential d
\DeclareMathOperator\sd{d\!} %differential d preceeded by a thin space, when needed
\newcommand{\ri}{\mathrm{i}} %imaginary i
\newcommand{\re}{\mathrm{e}} %exponential e
\DeclareMathOperator\sP{P}   %probability, treated as an operator
\newcommand{\rP}{\mathrm{P}} %probability, when spacing isn't a worry
\DeclareMathOperator\sE{E}   %expectation, treated as an operator
\newcommand{\rE}{\mathrm{E}} %expectation, when spacing isn't a worry
\DeclareMathOperator\BigO{O} %Big-0
\DeclareMathOperator\littleo{o} %little-o
\DeclareMathOperator\diag{diag}
%
\newcommand{\trans}{^\top}   %matrix transpose
%
\newcommand{\cvgpr}{\xrightarrow{\text{\upshape\tiny P}}}   %convergence in probability
\newcommand{\cvgas}{\xrightarrow{\text{\upshape\tiny a.s.}}}   %convergence almost surely
\newcommand{\cvgwk}{\xrightarrow{\text{\upshape\tiny W}}}   %weak convergence
\newcommand{\cvgdist}{\xrightarrow{\text{\upshape\tiny D}}} %convergence in distribution
\newcommand{\eqdist}{\stackrel{\text{\upshape\tiny D}}{=}}  %equality in distribution
\newcommand{\cvgs}{\stackrel{\text{\upshape\tiny S}}{\implies}}
\newcommand{\cvgmz}{\stackrel{\text{\upshape\tiny MZ}}{\implies}}

\DeclareMathOperator{\sgn}{sgn}
\DeclareMathOperator{\expint}{E_1}
\DeclareMathOperator{\Norm}{N}
\DeclareMathOperator{\order}{O}
\DeclareMathOperator{\ind}{\mathbf{1}}
\DeclareMathOperator{\besselK}{K}
\newcommand{\reals}{\mathbb{R}}
\newcommand{\plus}{{\mathord{+}}}
\newcommand{\minus}{{\mathord{-}}}
\newcommand{\naturalnumbers}{\mathbb{N}}
\newcommand{\integers}{\mathbb{Z}}
\newcommand{\esssup}{\mathop{\rm ess\,sup}}
\newcommand{\argmin}{\mathop{\rm arg\,min}}
\newcommand{\argmax}{\mathop{\rm arg\,max}}

\def\combin#1#2{\left(\begin{array}{c}\!\!{#1}\!\!\\\!\!{#2}\!\!\end{array}\right)}

\makeatletter
\def\imod#1{\allowbreak\mkern10mu({\operator@font mod}\,\,#1)}
\makeatother

\newtheorem{thm}{Theorem}[section]
%\newtheorem{cor}[thm]{Corollary}
%\newtheorem{lem}[thm]{Lemma}
%\newtheorem{prop}[thm]{Proposition}
%\newtheorem{conj}[thm]{Conjecture}
%\newtheorem{remark}[thm]{Remark}
\newtheorem{lem}{Lemma}[section]
\newtheorem{remark}{Remark}[section]

%\newenvironment{remark}[1][Remark]{\begin{trivlist}
%\item[\hskip \labelsep {\bfseries #1}]}{\end{trivlist}}

\newcounter{enumtracker}
\newcounter{enumtrackerone}

% Centré, avec un signe devant le numéro de appendix
\makeatletter
\def\appendix{\renewcommand{\thesection}{\thechapter.\Alph{section}}
	\@ifstar\unnumberedappendix\numberedappendix}
\def\numberedappendix{\@ifnextchar[%]
  \numberedappendixwithtwoarguments\numberedappendixwithoneargument}
\def\unnumberedappendix{\@ifnextchar[%]
  \unnumberedappendixwithtwoarguments\unnumberedappendixwithoneargument}
\def\numberedappendixwithoneargument#1{\numberedappendixwithtwoarguments[#1]{#1}}
\def\unnumberedappendixwithoneargument#1{\unnumberedappendixwithtwoarguments[#1]{#1}}
\def\numberedappendixwithtwoarguments[#1]#2{%
  \ifhmode\par\fi
  \removelastskip
  \vskip 2ex\goodbreak
  \refstepcounter{section}%
  \begingroup
  \noindent\leavevmode\Large\bfseries\raggedright 
  Appendix\ \thesection\quad#2\par
  \endgroup
  \vskip 1ex\nobreak
  \addcontentsline{toc}{section}{%
    \protect\numberline{\thesection}%
    #1}%
  }
\def\unnumberedappendixwithtwoarguments[#1]#2{%
  \ifhmode\par\fi
  \removelastskip
  \vskip 2ex\goodbreak
%  \refstepcounter{section}%
  \begingroup
  \noindent\leavevmode\Large\bfseries\raggedright 
%  \S \thesection\ 
  #2\par
  \endgroup
  \vskip 1ex\nobreak
  \addcontentsline{toc}{section}{%
%    \protect\numberline{\thesection}%
    #1}%
  }
\makeatother


